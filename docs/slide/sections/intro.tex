\section{Latar Belakang}
\subsection{Topik Penelitian}
\begin{frame}{Topik Penelitian}
  \begin{center}
    \Large ``Aplikasi \textit{Convolutional Neural Network} dalam Prediksi Distribusi Tekanan Udara di Permukaan Airfoil''
  \end{center}
\end{frame}

\subsection{Target Penelitian}
\begin{frame}{Target Penelitian}
  Aplikasi diharapkan dapat memprediksikan distribusi tekanan udara di permukaan airfoil ketika pengguna memberikan \textbf{geometri} dan \textbf{sudut serang} airfoil.
\end{frame}

\subsection{Luaran SK6091-01}
\begin{frame}{Luaran SK6091-01}
  Luaran dari \textit{Penelitian Mandiri 1} diantaranya:
  \begin{enumerate}
    \item Mengubah data \textbf{geometri} dan \textbf{sudut serang} airfoil menjadi data citra melalui pengolahan citra dengan \texttt{OpenCV}\footnote{\url{https://opencv.org/}}.
    \pause
    \item Menjadi basis data latih berupa citra airfoil dalam bentuk SDF \cite{oleynikova2016signed} dan aerodinamika airfoil.
  \end{enumerate}
\end{frame}
